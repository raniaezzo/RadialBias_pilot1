% This part is the latex header. It defines what kind of document this will be and 
% says which packages to use. Packages let you include different kind of formats
% and templates in the document. 
\documentclass[11pt]{article} % document type - other options are journal or book?
\usepackage[pdftex]{graphicx} % package to import figures
\usepackage{float} % used for placing figures - H
\usepackage{hyperref} % used for including links (hyper links)
\usepackage{enumerate} % used for making lists
\usepackage[margin=1cm]{geometry}
\usepackage{tabularx}
\usepackage{booktabs}
\usepackage{amsmath}
\usepackage[version=3]{mhchem} 
\usepackage{siunitx}

% math packages
\usepackage{amssymb}
\usepackage{amsmath}

\pagestyle{headings}
\topmargin -0.5in
\oddsidemargin 0.0in
\textwidth 6.5in
\textheight 9.0in

% declare the title, date and author
\title{Radial Bias Pilot 1}
\date{April 28, 2021}
\author{Rania Ezzo}

% This is where the actual document starts
\begin{document}
\maketitle
\tableofcontents


\section{Goal of Pilot 1}
To measure radial direction bias with 1D drifting gratings at 8 polar angle locations at 7 deg eccentricity. A total of 3 motion conditions will be tested, 2 radial (inwards and outwards) and 1 tangential (combined), to measure the performance differences between (1) centrifugal and centripetal motion directions, and (2) radial and tangential motion directions. 

\subsection{Parameters}
Eccentricity from central fixation: 7 degrees
\\
Locations tested (polar angle relative to fixation): 0-315 degrees in 45 degree increments
\\
Stimulus: sine wave gratings w/ 0.4 deg sigma gaussian mask
\\
Stimulus spatial frequency: 1 c/deg
\\
Stimulus drift speed: 8 deg/s
\\
Stimulus contrast: 50\% contrast per grating + gaussian mask
\\
Stimulus aperature diameter: 2.5 deg
\\
Black circular aperature was put onto screen to avoid perceptual artifacts from screen edges
\\
Number of subjects: 1-2

\subsection{Subject Instructions}
For each of the following trials, a fixation dot will appear on the screen. A drifting pattern will appear at some distance from the center. Your task is to determine whether the pattern is drifting clockwise or counterclockwise relative to the reference.
\\
Please remain fixated on the dot throughout the trials.
\\
Press the RIGHT ARROW for clockwise direction.
\\
Press the LEFT ARROW for counterclockwise direction.

\subsection{Experimental Design}
The pilot uses a 2AFC paradigm, within a block each trial includes a drifting grating presented at 1 of 4 possible positions, while the subject maintains fixation at the central dot. A method of constant stimuli is used which is set based on the performance of the training session (see Methods). The angular values added to the internal reference frame is chosen at random from the following constants [-8, -4, -2, -1, -0.5, 0.5, 1, 2, 4, 8] -- logarithmic spacing from 0.5 to 8. The observer must determine whether the direction of motion if clockwise or counterclockwise relative to the internal reference. The sequence of each trial for the 4 motion standards (specific to diagonal locations) at one location is depicted below:

\begin{figure}[H]
\centering % centers the figure
\includegraphics[scale=.4]{Images/Radial_sequence.png}
\\
\includegraphics[scale=.4]{Images/Tang_sequence.png}
\caption{Blue arrow represents the internal reference, the orange arrow represents an example of the direction at which the stimulus is presented (can be clockwise or counterclockwise to the blue arrow.}
\end{figure}

\subsection{Block sequence}
Four blocks were run, and each block corresponded to 1 of the 4 conditions being tested (tangential lower left motion, tangential upper right motion, radial upper left motion, radial lower right motion). The internal reference frames for each block is shown below:

\begin{figure}[H]
\centering % centers the figure
\includegraphics[scale=.4]{Images/Blocks.png}
\end{figure}

Prior to the actual experiment, the "standard" motion direction corresponding to that specific block will be showed to the observer to use as an internal reference. Then a training session is conducted to determine how much tilt is required to meet 75\% accuracy with staircase procedure (MLPest), and to allow subject to practice task with feedback. The estimated angular value to add/subtract to the standard to achieve 75\% performance of the clockwise/counterclockwise will be used to determine constants. For this pilot, constants [-8, -4, -2, -1, -0.5, 0.5, 1, 2, 4, 8] were chosen for all 8 blocks. Note positive and negative values for clockwise v. counterclockwise tilt.
\\
%Each block contains (2 locations with clockwise/counterclockwise motion) x 80 repetitions = 1,280 trials. All 4 full-blocks took 80 min. 
Each block contained 4 locations x 5 tilt values x 2 (clock v cc) x 20 repetitions = 800 trials. There are 8 blocks * 800 trials = 6400 total trials (3200 tang, 1600 radial-in, 1600 radial-out). Each full-block takes ~45 min; all 8 blocks took 360 min. 
\\
RE sequence of blocks 
\begin{enumerate}
\item diag-UL [angles: +- 0.5, 1, 2, 4, 8] (45 min)
\item card-HR [angles: +- 0.5, 1, 2, 4, 8] (45 min)
\item diag-LR [angles: +- 0.5, 1, 2, 4, 8] (45 min)
\item card-VL [angles: +- 0.5, 1, 2, 4, 8] (45 min)
\item diag-UR [angles: +- 0.5, 1, 2, 4, 8] (45 min)
\item card-HL [angles: +- 0.5, 1, 2, 4, 8] (45 min)
\item diag-LL [angles: +- 0.5, 1, 2, 4, 8] (45 min)
\item card-VU [angles: +- 0.5, 1, 2, 4, 8] (45 min)
\end{enumerate}

\newpage
\section{Data}
\subsection{RE Psychometric Fits (Cumulative normal)}
\begin{figure}[H]
\centering % centers the figure
\includegraphics[scale=.06]{Images/PF_RE_allcond.png}
\includegraphics[scale=.11]{Images/MeanSlopeError_ci_RE_allcond.png}
\includegraphics[scale=.11]{Images/MeanBiasError_ci_RE_allcond.png}
\includegraphics[scale=.06]{Images/PF_RE_oblique.png}
\includegraphics[scale=.11]{Images/MeanSlopeError_ci_RE_oblique.png}
\includegraphics[scale=.11]{Images/MeanBiasError_ci_RE_oblique.png}
\includegraphics[scale=.06]{Images/PF_RE_cardinal.png}
\includegraphics[scale=.11]{Images/MeanSlopeError_ci_RE_cardinal.png}
\includegraphics[scale=.11]{Images/MeanBiasError_ci_RE_cardinal.png}
\caption{RE new data (speed 8 deg/s) across 8 blocks that each contain 1 reference vector. Top row: All trials (combining cardinal and oblique blocks). Each point = (20 x 8 locations); Second row: subset of data in first row, including only the oblique motion directions (diagonal locations). Each point = (20 x 4 locations); Last row: subset of data in first row, including only the cardinal motion directions (cardinal locations). Each point = (20 x 4 locations). Positive bias = more counterclockwise responses. All means/confidence intervals were computed from samples of posterior distribution using Markov chain Monte Carlo method (from PAL\_PFHB\_fitModel.m) - 5000 samples, 3 chains.}
\end{figure}

\textbf{RE SENSITIVIY/SLOPE}
\\
Radial out beta = [cardinal \& diagonal directions =  0.58, cardinal = 0.72, diagonal = 0.48]
\\
Radial in beta = [cardinal \& diagonal directions =  0.58, cardinal = 0.73, diagonal = 0.48]
\\
Tangential beta = [cardinal \& diagonal directions =  0.48, cardinal = 0.63, diagonal = 0.40]
\\
\textbf{RE BIAS}
\\
Radial out alpha = [cardinal \& diagonal directions =  -0.66, cardinal = -0.52, diagonal = -0.82]
\\
Radial in alpha = [cardinal \& diagonal directions =  -0.31, cardinal = -0.43, diagonal = -0.16]
\\
Tangential alpha = [cardinal \& diagonal directions =  -0.25, cardinal = -0.48, diagonal = 0.02]

\newpage
\subsection{RE Polar Performance Plots (New vs. Old)}
\begin{figure}[H]
\centering % centers the figure
\includegraphics[scale=.18]{Images/polarplot_new.png}
\includegraphics[scale=.35]{Images/performance_polarplot.png}
\caption{Polar plots by performance (range 50-100\%). Each point is 400 trials (collapsed across blocks). LEFT: new data w/eyetracking. RIGHT: old data.}
\end{figure}
\subsection{Quality Control \& Misc.}
\begin{figure}[H]
\centering % centers the figure
\includegraphics[scale=.15]{Images/block_performance_new.png}
\includegraphics[scale=.15]{Images/block_performance.png}
\caption{To check that performance does not vary too much between blocks. Cardinal blocks are interweaved with diagonal blocks, and consistently show better performance. LEFT: new data w/eyetracking. RIGHT: old data.}
\end{figure}

\newpage
\section{Data}
\subsection{FH Psychometric Fits (Cumulative normal)}
\begin{figure}[H]
\centering % centers the figure
\includegraphics[scale=.06]{Images/PF_FH_allcond.png}
\includegraphics[scale=.11]{Images/MeanSlopeError_ci_FH_allcond.png}
\includegraphics[scale=.11]{Images/MeanBiasError_ci_FH_allcond.png}
\includegraphics[scale=.06]{Images/PF_FH_oblique.png}
\includegraphics[scale=.11]{Images/MeanSlopeError_ci_FH_oblique.png}
\includegraphics[scale=.11]{Images/MeanBiasError_ci_FH_oblique.png}
\includegraphics[scale=.06]{Images/PF_FH_cardinal.png}
\includegraphics[scale=.11]{Images/MeanSlopeError_ci_FH_cardinal.png}
\includegraphics[scale=.11]{Images/MeanBiasError_ci_FH_cardinal.png}
\caption{Same arrangement as previous page, but for subject FH (only includes half of full dataset: UR, VL, LL, HR).}
\end{figure}

\textbf{FH SENSITIVIY/SLOPE}
\\
Radial out beta = [cardinal \& diagonal directions =  0.29, cardinal = 0.62, diagonal = 0.20]
\\
Radial in beta = [cardinal \& diagonal directions =  0.28, cardinal = 0.70, diagonal = 0.22]
\\
Tangential beta = [cardinal \& diagonal directions =  0.27, cardinal = 0.41, diagonal = 0.20]
\\
\textbf{FH BIAS}
\\
Radial out alpha = [cardinal \& diagonal directions =  -0.61, cardinal = -0.59, diagonal = -0.49]
\\
Radial in alpha = [cardinal \& diagonal directions =  -0.27, cardinal = -1.39, diagonal = 1.66]
\\
Tangential alpha = [cardinal \& diagonal directions =  -0.48, cardinal = -0.78, diagonal = -0.03]

\newpage
\section{Data}
\subsection{BB Psychometric Fits (Cumulative normal)}
\begin{figure}[H]
\centering % centers the figure
\includegraphics[scale=.06]{Images/PF_BB_allcond.png}
\includegraphics[scale=.11]{Images/MeanSlopeError_ci_BB_allcond.png}
\includegraphics[scale=.11]{Images/MeanBiasError_ci_BB_allcond.png}
\includegraphics[scale=.06]{Images/PF_BB_oblique.png}
\includegraphics[scale=.11]{Images/MeanSlopeError_ci_BB_oblique.png}
\includegraphics[scale=.11]{Images/MeanBiasError_ci_BB_oblique.png}
\includegraphics[scale=.06]{Images/PF_BB_cardinal.png}
\includegraphics[scale=.11]{Images/MeanSlopeError_ci_BB_cardinal.png}
\includegraphics[scale=.11]{Images/MeanBiasError_ci_BB_cardinal.png}
\caption{Same arrangement as previous page, but for subject BB (only includes partial dataset: VU, LR). Not error range is too large (uninterpretable)}
\end{figure}

\textbf{BB SENSITIVIY/SLOPE}
\\
Radial out beta = [cardinal \& diagonal directions =  0.44, cardinal = 0.47, diagonal = 0.43]
\\
Radial in beta = [cardinal \& diagonal directions =  0.55, cardinal = 0.74, diagonal = 0.46]
\\
Tangential beta = [cardinal \& diagonal directions =  0.45, cardinal = 0.78, diagonal = 0.33]
\\
\textbf{FH BIAS}
\\
Radial out alpha = [cardinal \& diagonal directions =  0.37, cardinal = -0.21, diagonal = 0.95]
\\
Radial in alpha = [cardinal \& diagonal directions =  -1.60, cardinal = -1.14, diagonal = -2.1]
\\
Tangential alpha = [cardinal \& diagonal directions =  0.09, cardinal = 0.04, diagonal = 0.14]

\newpage
\section{Current Goals} 
\begin{enumerate}
	\item Several papers demontrate that sensitivity to radial orientations is greater than tangential orientations; similarly, radial direction bias is reported for moving dot stimuli. But orientation and motion direction is always orthogonal with 1D drifting gratings. Is sensitivity greater for radial motion or radial orientations w/ 1D drifting gratings? If the radial sensitivity is greater in respect to motion direction, then the radial orientation effect is weaker than the radial motion effects (or visa versa).
	\begin{itemize}
	\item{Interesting because the reported effects seem at odds, physiologically. Component neurons generally respond to motion that is orthogonal to their preferred orientation.} 
	\item{So far, data points to radial bias in respect to motion domain (conflict w/ Hong).}
	\end{itemize}
	\item Is sensitivity generally greater in cardinal locations compared to non-cardinal locations?
	\begin{itemize}
	\item{Note: if sensitivity is higher in cardinal locations, this could be due to the location or due to the feature of the stimulus (cardinality in orientation/motion).}
	\end{itemize}
	\item Is the difference in radial and tangential sensitivity more pronounced in non-cardinal locations compared to cardinal locations?
	\begin{itemize}
	\item{Note: Cardinal bias might enhance sensitivity disproportionately for orientation, which minimizes radial bias differences on the cardinal axes.}
	\end{itemize}
	\end{enumerate}
	\begin{figure}[H]
	\centering % centers the figure
	\includegraphics[scale=.25]{Images/Cartoon1.png}
	\end{figure}
	\begin{figure}[H]
	\centering % centers the figure
	\includegraphics[scale=.25]{Images/Cartoon2.png}
	\includegraphics[scale=.25]{Images/Cartoon3.png}
	\end{figure}
	\begin{figure}[H]
	\centering % centers the figure
	\includegraphics[scale=.25]{Images/Cartoon4.png}
	\end{figure}
	
\newpage
\section{Extending to Plaid Stimuli} 
\begin{enumerate}
	\item How does this extend to plaid stimuli? Does the bias apply to the component motion direction or the perceived motion direction?
	\begin{figure}[H]
	\centering % centers the figure
	\includegraphics[scale=.25]{Images/Cartoon5.png}
	\end{figure}
\end{enumerate}
\section{Supplemental Questions} 
\begin{enumerate}
	\item Is there an HVA or VMA present for any/all of the conditions? (SF in design matters in this case)
	\item Is there a difference in sensitivity to radial-inward vs. radial-outwards motion? 
	\begin{itemize}
	\item{So far, there doesn't seem to be a difference at 7 deg eccentricity.}
	\end{itemize}
	\item Maybe abandon question about how these biases change w/ eccentricity.
\end{enumerate}

\newpage
\section{Updates} 
\begin{itemize}
\item Design Related
	\begin{itemize}
	\item Aborts trials that have breaks in fixation during stimulus presentation (stimulus\_start - 300 ms TO stimulus\_end)
	\item Changed contrast to 50\% -- this should not affect difficulty in motion perception
	\end{itemize}
\item Analysis Related
	\begin{itemize}
	\item n/a
	\end{itemize}
\item Other improvements
	\begin{itemize}
	\item n/a
	\end{itemize}
\item For discussion
	\begin{itemize}
	\item Eye tracker in RM 956
		\begin{itemize}
			\item{Afp server not compatible with PC? -- where to backup data?}
			\item{CRT monitor calibration.}
		\end{itemize}
	\item No longer traveling to AD (July 18 - Late Aug)
	\item VSS expenses
	\end{itemize}
\end{itemize}

\section{To Do} 
\begin{itemize}
\item Feedback
	\begin{itemize}
	\item Fix beep volume (lower pitch sounds lower in volume)
	\item Fix aperture to ensure no sides show for different heights
	\item Ensure speed is the same across locations (Billy \& Jon both mention some seem faster)
	\item Can I change response time to happen sooner?
	\item Type up instructions for experiment
	\end{itemize}
\item Other
	\begin{itemize}
	\item Re-plot polar angle plot with arrows pointing in direction (length indicating performance); change to dprime
	\item Collect data with current params for 1-2 subjects w/ eyetracker
	\item If we want to capture polar angle differences, might need to increase SF? (confirmed in other exp around 6 cpd, 6 deg ecc)
	\item Titration for cardinal v. oblique? Or keep the constant stimuli? Dynamic staircase methods?
	\item Report sensitivity (d prime) instead of performance
	\item Maybe report reliability? \begin{equation}1/sigma^{2}\end{equation}
	\item Double check sigma of gaussian for reporting purposes (and at what eccentricity contrast drops below 1 perc)
	\end{itemize}
\end{itemize}

\section{Software to Cite}
\begin{itemize}
\item PsychToolbox Extensions (Brainard, 1997; Pelli, 1997; Kleiner et al, 2007)
\item Prins, N \& Kingdom, F. A. A. (2018) Applying the Model-Comparison Approach to Test Specific Research Hypotheses in Psychophysical Research Using the Palamedes Toolbox. Frontiers in Psychology, 9:1250. doi: 10.3389/fpsyg.2018.01250
\item Plummer, M. (2003, March). JAGS: A program for analysis of Bayesian graphical models using Gibbs sampling. In Proceedings of the 3rd international workshop on distributed statistical computing (Vol. 124, No. 125.10, pp. 1-10). (http://mcmc-jags.sourceforge.net/)
\end{itemize}

\section{Supplementary Images}
\subsection{FH}
\begin{figure}[H]
\centering % centers the figure
\includegraphics[scale=.25]{Images/FH_trialdescription.png}
\includegraphics[scale=.15]{Images/block_performance_FH.png}
\\
\includegraphics[scale=.15]{Images/FH_cardinal_plot.png}
\includegraphics[scale=.15]{Images/FH_oblique_plot.png}
\end{figure}
\subsection{BB}
\begin{figure}[H]
\centering % centers the figure
\includegraphics[scale=.25]{Images/BB_trialdescription.png}
\includegraphics[scale=.15]{Images/block_performance_BB.png}
\\
\includegraphics[scale=.15]{Images/BB_cardinal_plot.png}
\includegraphics[scale=.15]{Images/BB_oblique_plot.png}
\end{figure}

\end{document}